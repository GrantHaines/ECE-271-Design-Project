\documentclass[a4paper]{article}

%% Language and font encodings
\usepackage[english]{babel}
\usepackage[utf8x]{inputenc}
\usepackage[T1]{fontenc}
\usepackage[section]{placeins}
\usepackage{graphicx}
\usepackage{caption}
\usepackage{subcaption}

\usepackage{fancyvrb}	

%\usepackage{pdfpages}%to insert pdfs
 
\usepackage{float}		%Ben trying to format figures

%% Sets page size and margins
\usepackage[a4paper,top=3cm,bottom=2cm,left=3cm,right=3cm,marginparwidth=1.75cm]{geometry}

%% Useful packages
\usepackage{amsmath}
\usepackage{graphicx}
\usepackage[colorinlistoftodos]{todonotes}
\usepackage[colorlinks=true, allcolors=blue]{hyperref}
\makeatletter

%% All below Added by Ben |||||||||||||||||||||||||||||||||||||||||||||||
\g@addto@macro\@floatboxreset\centering
\makeatother				% this and above automatically center figures

% Alphabetized list macro from:  https://tex.stackexchange.com/questions/121489/alphabetically-display-the-items-in-itemize
\usepackage{datatool}% http://ctan.org/pkg/datatool
\newcommand{\sortitem}[1]{%
  \DTLnewrow{list}% Create a new entry
  \DTLnewdbentry{list}{description}{#1}% Add entry as description
}
\newenvironment{sortedlist}{%
  \DTLifdbexists{list}{\DTLcleardb{list}}{\DTLnewdb{list}}% Create new/discard old list
}{%
  \DTLsort{description}{list}% Sort list
  \begin{itemize}%
    \DTLforeach*{list}{\theDesc=description}{%
      \item \theDesc}% Print each item
  \end{itemize}%
}
%use these instead to alphabetize list:
%\begin{sortedlist}
%   \sortitem{ISDYNSTP
%
%||||||||||||||||||||||||||||||||||||||||||||||||||||||||||||||||||||||||||


\title{ECE271, Final Project}
\author{Ben Adams, Grant Haines, Benjiman Walsh}
\date{\today}

\begin{document}
\maketitle

\pagebreak

\tableofcontents

\section{Introduction}

The purpose of this project is to create a digital logic design that uses an NES controller to control various kinds of output through an FPGA.



\section{High Level Descriptions}%%%%%%%%%%%%%%



\section{Controller Descriptions}



\section{HDL Components}



\section{Appendix}

\subsection{Source Code}%%%%%%%%%%%%%%%%%%%%%%%%%%%%%%%%%%%

\subsubsection{NES Controller Reader}

\subsubsection{Square Wave Generator}

\begin{Verbatim}[tabsize=4]
module periodTime(input logic clk,
					input logic [2:0] data,
					output logic q);
					
int compareNumber;
int count;

always_comb
		case(data)
			0: compareNumber = 6400;	// just mod input clock until audio spectrum periods
			1: compareNumber = 3200;	// one octave
			2: compareNumber = 1600;
			3: compareNumber = 8000;
			
			4: compareNumber = 4000;
			5: compareNumber = 2000;
			6: compareNumber = 1000;
			7: compareNumber = 500;		// consider adding default case
		endcase
			
always_ff @(posedge clk)
	begin
		if (count >= compareNumber)	// could modify to not restart notes when changed
			count <= 0;
		else 
			count <= count +1;
	end

always_comb
	begin
			if( count < compareNumber)				
				q = (count > compareNumber/2); 	//assigns output with initial low
			else 
				q = 0;
	end

endmodule
\end{Verbatim}

\subsection{Simulation Results}%%%%%%%%%%%%%%%%%%%%%%%%%%%%%

\subsubsection{NES Controller Reader}
\begin{figure}[H]
    \includegraphics[width=0.8 \linewidth]{images/NESSIM.png}
    \caption{"Button Mashing" on the NES}
    \label{nesButtonMash}
\end{figure}

At first, I wanted to test the NES controller reader by just simulating a bunch of random inputs as seen in Figure \ref{nesButtonMash}. I remembered the NES game CONTRA had a cheat code that involved most of the controller's buttons (all but SELECT). The "Contra Code" was then simulated I'm gonna rewrite this.
 
\begin{figure}[H]
    \includegraphics[width=0.8 \linewidth]{images/contra.jpg}
    \caption{CONTRA screenshot}
    \label{contra}
\end{figure}

\begin{figure}[H]
    \includegraphics[width=0.8 \linewidth]{images/NESSIMcontra.png}
    \caption{Simulating the "Contra Code" }
    \label{contraSim}
\end{figure}

\begin{verbatim}

force -freeze sim:/NesReader/dataYellow 0 0, 0 {20 ps} , 0 {40 ps} , 0 {60 ps} , 
    1 {80 ps} , 0 {100 ps} , 0 {120 ps} , 0 {140 ps}	#up
force -freeze sim:/NesReader/dataYellow 0 0, 0 {20 ps} , 0 {40 ps} , 0 {60 ps} , 
    0 {80 ps} , 1 {100 ps} , 0 {120 ps} , 0 {140 ps}	#down
force -freeze sim:/NesReader/dataYellow 0 0, 0 {20 ps} , 0 {40 ps} , 0 {60 ps} , 
    1 {80 ps} , 0 {100 ps} , 0 {120 ps} , 0 {140 ps}	#up
force -freeze sim:/NesReader/dataYellow 0 0, 0 {20 ps} , 0 {40 ps} , 0 {60 ps} , 
    0 {80 ps} , 1 {100 ps} , 0 {120 ps} , 0 {140 ps}	#down
force -freeze sim:/NesReader/dataYellow 0 0, 0 {20 ps} , 0 {40 ps} , 0 {60 ps} , 
    0 {80 ps} , 0 {100 ps} , 1 {120 ps} , 0 {140 ps}	#left
force -freeze sim:/NesReader/dataYellow 0 0, 0 {20 ps} , 0 {40 ps} , 0 {60 ps} , 
    0 {80 ps} , 0 {100 ps} , 0 {120 ps} , 1 {140 ps}	#right
force -freeze sim:/NesReader/dataYellow 0 0, 0 {20 ps} , 0 {40 ps} , 0 {60 ps} , 
    0 {80 ps} , 0 {100 ps} , 1 {120 ps} , 0 {140 ps}	#left
force -freeze sim:/NesReader/dataYellow 0 0, 0 {20 ps} , 0 {40 ps} , 0 {60 ps} , 
    0 {80 ps} , 0 {100 ps} , 0 {120 ps} , 1 {140 ps}	#right
force -freeze sim:/NesReader/dataYellow 0 0, 1 {20 ps} , 0 {40 ps} , 0 {60 ps} , 
    0 {80 ps} , 0 {100 ps} , 0 {120 ps} , 0 {140 ps}	#b	
force -freeze sim:/NesReader/dataYellow 1 0, 0 {20 ps} , 0 {40 ps} , 0 {60 ps} , 
    0 {80 ps} , 0 {100 ps} , 0 {120 ps} , 0 {140 ps}	#a
force -freeze sim:/NesReader/dataYellow 0 0, 0 {20 ps} , 0 {40 ps} , 1 {60 ps} , 
    0 {80 ps} , 0 {100 ps} , 0 {120 ps} , 0 {140 ps}	start
\end{verbatim}

\subsubsection{Square Wave Generator}

\begin{figure}[H]
    \includegraphics[width=0.8 \linewidth]{images/squareSim.png}
    \caption{Simulating button inputs to control the square wave oscillator}
    \label{squareSim}
\end{figure}


%\includepdf[page=-]{Ch3Example}

\end{document}
